\pagestyle{empty}
\cleardoublepage
\pagenumbering{gobble}
\chapter*{Checkliste Abschlussarbeit}
In der folgenden Liste sind die wichtigsten, häufig nachgefragten Punkte rund um Ihre Abschlussarbeit zum Abhaken zusammengefasst.

\noindent{\color{red}Dieses Blatt ist nicht Bestandteil der Abschlussarbeit! Entfernen Sie den Include der \texttt{Checklist.tex} aus der \texttt{thesis.tex} bevor Sie die Ausarbeitung in den Druck geben.}

\begin{itemize}
    \item[$\square$]{
        formale Kriterien mit beiden Begutachtenden klären
        \begin{itemize}
            \item[$\square$] Einseitig oder doppelseitig?
            \item[$\square$] Zitationsstil und Referenzierstil?
            \item[$\square$]{
                Sonstige Vorgaben?\\
                (z.\,B. minimale Schriftgröße in Abbildungen)
            }
        \end{itemize}
    }
    \item[$\square$]{
        Anzahl Print-Exemplare klären
        \begin{itemize}
            \item[$\square$] Pflicht-Exemplar für Archiv (\emph{muss} ein Print-Exemplar sein)
            \item[$\square$] begutachtende Person beim Praxispartner
            \item[$\square$] begutachtende Person an der Berufsakademie
        \end{itemize}
    }
    \item[$\square$]{
        Form der elektronischen Abgabe klären\\
        (die Prüfungsordnung macht hier keine Vorgabe!)
        \begin{itemize}
            \item[$\square$] Pflicht-Exemplar für Archiv (z.\,B. selbst gebrannte DVD)
            \item[$\square$] begutachtende Person beim Praxispartner
            \item[$\square$] begutachtende Person an der Berufsakademie
        \end{itemize}
    }
    \item[$\square$]{
        Auftragsblatt
        \begin{itemize}
            \item[$\square$] Auftragsblatt einscannen
            \item[$\square$] Original des Auftragsblatt in ein Print-Exemplar einbinden
            \item[$\square$] Kopie des Auftragsblatts in die anderen Print-Exemplare einbinden
            \item[$\square$] Scan des Auftragsblatts in elektronischer Version einfügen
        \end{itemize}
    }
    \item[$\square$]{
        alle referenzierten Quellen auf Quelleneigenschaften prüfen\\
        (alles, was nicht den Quelleneigenschaften genügt, sollten Sie in Fußnoten auslagern)
    }
    \item[$\square$]{
       Notwendigkeit des Thesenblatts klären\\
       (weder in der Prüfungsordnung noch dem Modulhandbuch existiert das Thesenblatt)
    }
    \item[$\square$]{
        Zusammenfassung der Arbeit mit allen wichtigen Ergebnissen schreiben
        \begin{itemize}
            \item[$\square$] geschriebene Zusammenfassung in Abstract übernehmen
            \item[$\square$] geschriebene Zusammenfassung leicht abwandeln und in Thesenblatt übernehmen
        \end{itemize}
    }
    \item[$\square$]{
        Forschungsfragen und Thesen
        \begin{itemize}
            \item[$\square$] Werden die Forschungsfragen tatsächlich in der Ausarbeitung beantwortet?
            \item[$\square$] Falls keine Forschungsfragen: Werden die Hypothesen tatsächlich in der Ausarbeitung bestätigt oder widerlegt? Gibt es bei widerlegten Thesen Antithesen und Synthesen?
            \item[$\square$]{
                Wurden aus den Ergebnissen der wissenschaftlichen Untersuchung Thesen abgeleitet?
                \begin{itemize}
                    \item[$\square$] Forschungsfragen auf Thesenblatt übertragen
                    \item[$\square$] Thesen auf Thesenblatt übertragen
                \end{itemize}
            }
        \end{itemize}
    }
\clearpage
    \item[$\square$]{
        Druck und Abgabe vorbereiten
        \begin{itemize}
             \item[$\square$]{
                Beim Druckgeschäft des Vertrauens nach Druckkapazitäten fragen\\
                (Neben Ihnen wollen viele andere Studierende ihre Abschlussarbeiten drucken lassen!)
                \begin{itemize}
                    \item[$\square$] Kapazitäten in der Woche der Abgabe?
                    \item[$\square$] Bis wann spätestens PDF im Geschäft abgeben, damit Wunsch-Fertigstellungstermin gehalten werden kann?
                    \item[$\square$] Kosten Graustufen- vs. Farbdruck klären
                    \item[$\square$]{
                        archivierbare Bindung klären\\
                        (Das Pflichtexemplar für das Archiv der Berufsakademie muss langzeit-archivierbar sein und darf keine Klebstoffe in der Bindung enthalten!)
                    }
                \end{itemize}
            }
            \item[$\square$]{
                Im Service-Büro nachfragen:
                \begin{itemize}
                    \item[$\square$] Bürozeiten und/oder Vertretungsregelung, falls Sie persönlich abgeben wollen
                    \item[$\square$]{
                        korrekte Adressierung für postalische Einreichung\\
                        (Beachten Sie, dass die Deutsche Post AG lediglich Zustellungen binnen $3$ Werktagen für $95\,\%$ der Sendungen garantiert!)\\
                        (Einschreiben und Rückschein bringen juristisch nichts, da diese nur nachweisen, dass Sie \emph{etwas} eingesendet haben, aber nicht \emph{was}. Wenn Sie rechtssicher auch den Inhalt der Einsendung, nämlich die Print-Exemplare, nachweisen wollen, müssen Sie einen Gerichtsvollzieher beauftragen! Achtung: Teuer.)
                    }
                \end{itemize}
            }
            \item[$\square$]{
                Unterschriften und Daten prüfen
                \begin{itemize}
                    \item[$\square$]{
                        Abgabedatum korrekt auf Abschlussarbeit?
                        \begin{itemize}
                            \item[$\square$] Vor oder gleich spätester Abgabe laut Auftragsblatt bzw. genehmigter Verlängerung?
                            \item[$\square$] Bei genehmigter Verlängerung: ursprüngliches spätestes Abgabedatum laut Auftragsblatt in Klammern hinter Abgabedatum?
                        \end{itemize}
                    }
                    \item[$\square$] Datum der Erklärung an Eidesstatt vor oder gleich Abgabedatum?
                    \item[$\square$]{
                        Erklärung an Eidesstatt unterschrieben?
                        \begin{itemize}
                            \item[$\square$] Print-Exemplar(e)
                            \item[$\square$] PDF-Datei
                        \end{itemize}
                    }
                \end{itemize}
            }
        \end{itemize}
    }
    \item[$\square$]{
        Verteidigung vorbereiten
        \begin{itemize}
            \item[$\square$]{
                Begutachtende nach Kritikpunkten und Fragen aus den Gutachten fragen\\
                (die Gutachten müssen laut Prüfungsordnung binnen $4$ Wochen nach Abgabe vorliegen)\\
                (laut Rahmengesetz haben Sie sogar einen Anspruch auf Einsicht in die Gutachten (Einsicht, nicht Kopie/Scan!), spätestens $2$ Wochen vor dem Verteidigungstermin; im Moment ignoriert die Berufsakademie diesen Rechtsanspruch, also formulieren Sie das Ganze als freundliche Anfrage an die Begutachtenden)
                \begin{itemize}
                    \item[$\square$] Gutachten Praxispartner
                    \item[$\square$] Gutachten Berufsakademie
                \end{itemize}
            }
            \item[$\square$]{
                Auf Basis der Kritikpunkte und Fragen ggf. inhaltlich nachbessern.\\
                (Sie sollten in der Verteidigung gezielt auf Kritik eingehen und Verbesserungen präsentieren, z.\,B. \enquote{Im Gutachten wurde die Nutzerstudie in der Evaluation bemängelt. Ich habe inzwischen eine systemisch verbesserte Befragung durchgeführt; die Ergebnisse konnten im Rahmen des Fehlermaßes reproduziert und somit bestätigt werden.})
            }         
            \item[$\square$]{
                Notwendigkeit des Posters für die Verteidigung klären\\
                (weder in der Prüfungsordnung noch im Modulhandbuch existiert das Poster)
            }
            \item[$\square$]{
                Möglichkeit der Prüfung der Präsentation vor der Verteidigung klären
                \begin{itemize}
                    \item[$\square$] Kollegium, Freundeskreis, Familie, $\ldots$
                    \item[$\square$] begutachtende Person beim Praxispartner
                    \item[$\square$] begutachtende Person an der Berufsakademie
                \end{itemize}
            }
        \end{itemize}
    }
\end{itemize}