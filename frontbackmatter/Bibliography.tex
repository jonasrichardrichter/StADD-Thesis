\renewcommand{\bibname}{Quellenverzeichnis}
\bibliographystyle{baalphadin}

\pdfbookmark[-1]{Quellenverzeichnis}{pdf-Bibliography}
\bibliography{bibliography}\label{Bibliography}
% BibLateX:
% \printbibliography
\addcontentsline{toc}{chapter}{Quellenverzeichnis}

\clearpage\section*{Hilfsmittel}\label{Tools}
Zur Erstellung der vorliegenden Abschlussarbeit wurden die folgenden Hilfsmittel verwendet:
% Seien Sie so spezifisch wie möglich!
\begin{itemize}
    \item ChatGPT --\,\url{https://chat.openai.com/chat}
	\item Comprehensive \TeX Archive Network, The --\,\url{http://ctan.org},
	\item dict.leo.org by LEO GmbH,
	\item Encyclop{\ae}dia Britannica,
	\item Google-Scholar,
	\item Korrekturlesende (namentlich: Erika und Franz Mustermensch),
	\item Microsoft Office 365 University (2107 Build 14228.20204, x64),
	\item Microsoft Visio Professional 2019 (2107 Build 14228.20204, x64),
	\item Microsoft Windows 10 (22H2 Build 19045.2486, x64)
	\item Overleafwin,
	\item Paint.NET (5.0.1, x86),
	\item S{\"a}chsische Landesbibliothek \mbox{--\,Staats}- und Universit{\"a}tsbibliothek Dresden, und
	\item Wiktionary.
\end{itemize}

\comment{
    Sie sollten bei Software eine möglichst abschließende und präzise Liste (inkl. Versionsnummer, etc.) angeben.
}