\renewcommand{\bibname}{Quellenverzeichnis}
\bibliographystyle{baalphadin}

\pdfbookmark[-1]{Quellenverzeichnis}{pdf-Bibliography}
\bibliography{bibliography}\label{Bibliography}
% BibLateX:
% \printbibliography
\addcontentsline{toc}{chapter}{Quellenverzeichnis}

\clearpage\section*{Hilfsmittel}
Zur Erstellung der vorliegenden Abschlussarbeit wurden die folgenden Hilfsmittel verwendet:
% Seien Sie so spezifisch wie möglich!
\begin{itemize}
	\item Adobe Acrobat Pro DC (2021.005.20060, x64),
	\item Adobe Dreamweaver 2021 (21.1 Build 15413, x64),
	\item Black Magic DaVince Resolve (17.2.2, x64),
	\item Comprehensive \TeX Archive Network, The --\,\url{http://ctan.org},
	\item CorelDRAW X7 (17.1.0.572, x86),
	\item Corel PHOTO-PAINT X7 (17.1.0.572, x86),
	\item dict.leo.org by LEO GmbH,
	\item Encyclop{\ae}dia Britannica,
	\item Google-Scholar,
	\item HeidiSQL (11.3.0.6337,x64 portable),
	\item KiTTY (0.74.4.13 Classic),
	\item Korrekturlesende (namentlich: Erika und Franz Mustermensch),
	\item Microsoft Office 365 University (2107 Build 14228.20204, x64),
	\item Microsoft Visio Professional 2019 (2107 Build 14228.20204, x64),
	\item Microsoft Windows (21H1 Build 19043.1110, x64)
	\item MiK\TeX\ (2.9.5846, x64),
	\item Paint.NET (4.2.16.7781.39227, x86),
	\item S{\"a}chsische Landesbibliothek \mbox{--\,Staats}- und Universit{\"a}tsbibliothek Dresden, und
	\item Wiktionary.
\end{itemize}

\comment{
    Sie sollten bei Software eine möglichst abschließende und präzise Liste (inkl. Versionsnummer, etc.) angeben.
}