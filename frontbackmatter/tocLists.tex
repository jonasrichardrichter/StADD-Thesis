\doparttoc
\dopartlof
\dopartlot

\makeatletter
\renewcommand\@pnumwidth{1.25cm}
\makeatother
\cleardoublepage\pdfbookmark{\contentsname}{toc}\tableofcontents\label{refTOC}
\cleardoublepage

\renewcommand{\listtheoremname}{Definitionen, Theoreme \& Beweise}
\cleardoublepage\phantomsection
\renewcommand\thmtformatoptarg[1]{: \enquote{#1}}
\addcontentsline{toc}{chapter}{Definitionen, Theoreme und Beweise}\listoftheorems[numwidth={3.7em},ignoreall,onlynamed={mydef,mytheorem,myproof}]

\renewcommand{\listfigurename}{Abbildungen}
\cleardoublepage\phantomsection
\makeatletter
	\let\org@dottedtocline\@dottedtocline
	\begingroup
		\renewcommand*\@dottedtocline[3]{\org@dottedtocline{#1}{#2}{3.7em}}
		\addcontentsline{toc}{chapter}{Abbildungsverzeichnis}\listoffigures
	\endgroup
	\renewcommand*\@dottedtocline[3]{\org@dottedtocline{#1}{#2}{#3}}
\makeatother

\renewcommand{\listtablename}{Tabellen}
\cleardoublepage\phantomsection
\makeatletter
	\begingroup
		\renewcommand*\@dottedtocline[3]{\org@dottedtocline{#1}{#2}{3.7em}}
		\addcontentsline{toc}{chapter}{Tabellenverzeichnis}\listoftables
	\endgroup
	\renewcommand*\@dottedtocline[3]{\org@dottedtocline{#1}{#2}{#3}}
\makeatother
\cleardoublepage

\renewcommand{\lstlistlistingname}{Quellkode}
\cleardoublepage\phantomsection
\makeatletter
	\begingroup
		\renewcommand*\@dottedtocline[3]{\org@dottedtocline{#1}{#2}{3.7em}}
		\addcontentsline{toc}{chapter}{Quellkodeverzeichnis}\lstlistoflistings
	\endgroup
	\renewcommand*\@dottedtocline[3]{\org@dottedtocline{#1}{#2}{#3}}
\makeatother

% Glossar
\cleardoublepage\phantomsection
\makeatletter
	\begingroup
		\renewcommand*\@dottedtocline[3]{\org@dottedtocline{#1}{#2}{3.7em}}
		\addcontentsline{toc}{chapter}{Glossar}\printglossary
% Wenn Sie keine Seitenangaben für Rücksprünge haben wollen,
% können diese mit der Option „nonumberlist“ unterdrückt werden:
% \printglossary[nonumberlist]
	\endgroup
	\renewcommand*\@dottedtocline[3]{\org@dottedtocline{#1}{#2}{#3}}
\makeatother

\cleardoublepage