\addcontentsline{toc}{chapter}{Zusammenfassung}
\section*{Zusammenfassung}
%
% Autorenreferat
%
% Hinweis: Autorenreferate sind in Ingenieurwissenschaften eher
% unüblich. Deshalb in dieser Vorlage standardmäßig deaktiviert.
%
\ifthenelse{\boolean{extendedabstract}}{
	{\myName}: \emph{\myTitle\ifdef{\mySubtitle}{\ --\,\mySubtitle}{}}. {\myCompany}. Berufsakademie Sachsen, {\myAcademy}, {\myStudy}, Bachelor-Arbeit, {\mySubmissionDate}.
	
	\medskip
	
	\number\numexpr\getpagerefnumber{contentPages:End}-\getpagerefnumber{chap:Intro}\relax\ Seiten\hfill \arabic{citenum}\ Literaturquellen\hfill  \total{chapter}\ Anhänge

	\hrulefill
	\medskip
}{}

Kurze Zusammenfassung des Inhaltes in deutscher Sprache; maximal eine halbe Seite lang. Die Zusammenfassung ist dabei ein einzelner Absatz (im Sinne des \LaTeX-Quelltextes: Die Zusammenfassung enthält keine doppelten Zeilenumbrüche). \textbf{Sie fasst die ganze Arbeit zusammen} und präsentiert auch bereits die wichtigsten Ergebnisse. Lesende sollen auf Basis der Zusammenfassung entscheiden können, ob es sich lohnt, die ganze Arbeit zu lesen oder nicht. Fehlen die wichtigsten Informationen (also insbesondere die Ergebnisse), wird die Arbeit eher nicht gelesen werden\dots\ \mbox{--\,Soll} oberhalb der Zusammenfassung ein Autorenreferat gesetzt werden, weisen Sie der Variable \texttt{extendedabstract} den Wert \texttt{true} in der Konfiguration im Punkt \emph{2a} in der \texttt{config.tex} zu.

%
% Bei Verwendung von Part (Teil) nicht entfernen!
% Dies sorgt dafür, dass die PDF-Lesezeichen in der richtigen
% Hierarchieebene liegen.
\pdfbookmark[-1]{Verzeichnisse}{pdf-Tables}