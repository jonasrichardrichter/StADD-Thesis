\chapter{Einleitung}\label{chap:Intro}
Der Leitfaden der Staatlichen Studienakademie Dresden für Abschlussarbeiten\footnote{\label{fn:Leitfaden}Im Genauen: \enquote{Verbindlicher Leitfaden für die Anfertigung und formale Gestaltung wissenschaftlicher (Haus-)Arbeiten an der der Staatlichen Studienakademie Dresden}} ist für Anforderungen in den Ingenieurwissenschaften stellenweise ungeeignet oder widerspricht grundsätzlich den Gepflogenheiten im Fachgebiet. Diese Vorlage hat das Ziel, Studierenden in den Ingenieurwissenschaften eine für das Fachgebiet passendere Vorlage anzubieten.

Diese Vorlage ist bei Overleaf\footnote{\url{https://www.overleaf.com/read/bfmdxnhntbxp} --\,zuletzt am 8. Februar 2023 erfolgreich abgerufen} und Github\footnote{\url{https://github.com/tchara/StADD-Thesis.git} --\,zuletzt am 8. Februar 2023 erfolgreich abgerufen} öffentlich verfügbar.

\comment{Beachten Sie bitte, dass dieses Dokument eine Empfehlung mit ein paar Leitlinien darstellt.\newline Es handelt sich \textbf{nicht} um eine verbindliche Richtlinie.}

\noindent Allgemein gelten die folgenden Grundsätze bei ingenieurwissenschaftlichen Ausarbeitungen:
\begin{itemize}
    \item{
        Ingenieure wollen Ressourcen sparen. Entsprechend sollten Sie sparsam mit Ihrem Textsatz umgehen. Dies bedeutet insbesondere:
        \begin{itemize}
            \item Doppelseitiges Layout (wenn Sie explizit einseitiges Layout wünschen, müssen Sie in der Datei \texttt{thesis.tex} in Zeile $30$ \enquote{twoside} auskommentieren),
            \item Fußnoten wieder verwenden (entweder \enquote{global} oder pro Kapitel),
            \item vollständige Quellenangaben nur im Literaturverzeichnis (im Hauptteil nur Verweise verwenden) und
            \item Sie als Autor dürfen eigene Inhalte kommentieren, wenn Sie die Lesenden beim Verstehen der Inhalte unterstützen wollen. Machen Sie Kommentare so deutlich kenntlich, dass sie sich eindeutig vom restlichen Fließtext unterscheiden! Siehe Beispiel oben.
            \item Der vorherige Punkt ist von Belang, da Sie sich nicht selbst plagiieren können; eigene Vorarbeiten, etc. können Sie nicht als Quellen verwenden! Um Lesende dennoch auf die Existenz von Vorarbeiten hinzuweisen, können Sie einen Kommentar von der Form \enquote{Ergebnisse aus der vorherigen Studienarbeit} o.\,ä. verwenden.
        \end{itemize}
    }
    \item{
        Quellen gehören in das Literaturverzeichnis am Ende (in dieser Vorlage auf Seite~\pageref{Bibliography}). Dabei führen Sie nur Primärquellen auf; wichtig dabei ist, dass als Quelle nur qualifizieren:
        \begin{itemize}
            \item Werke mit mindestens einer namentlich benannten Autor:in, einem Titel und einem Veröffentlichungsdatum (i.\,d.\,R. reicht das Jahr),
            \item Standarde (DIN, ISO, RFC, $\ldots$) oder
            \item \enquote{Berichte aus erster Hand} (zum Beispiel ein Vernehmungsprotokoll, eine Aussage vor Gericht, eine Zeitungsinterview, et cetera).
        \end{itemize}
    }
    \item{
        Insbesondere im Hauptteil der Ausarbeitung --\,in der Regel aber im gesamten Dokument\,-- referenzieren Sie die Quellen im Fließtext/Lauftext indem Sie auf den zitierten Autor\footnote{Das machen Sie mit dem Zitationsstil \enquote{unsrtnat} und der Anweisung \inlinecode{\textbackslash{}citeauthor\{quelle\}}.} oder die Referenz (z.B.\cite{bentley:1999}\footnote{Das hier ist ein spezieller Zitationsstil (\enquote{baalphadin}) für die Berufsakademie, welcher auf dem alphanummerischen DIN-Stil (\enquote{alphadin}) beruht. Sie können aber auch jeden anderen Stil verwenden, solange Sie eindeutig bleiben und nicht zwischen Stilen hin-und-her wechseln. Eine Liste mit Stilen finden Sie in der Overleaf-Hilfe: \url{https://www.overleaf.com/learn/latex/bibtex_bibliography_styles}}) im Quellenverzeichnis verweisen. {\color{red}In den Ingenieurwissenschaften gehören Quellenangaben auf keinen Fall in  Fußnoten!} Das bemerken Sie allein schon an den üblichen Zitierstilen aller großen Outlets, welche in den Ingenieur- und Naturwissenschaften verwendet werden: ACM, APA, CSE, DIN ISO 690 (Vancouver), IEEE, LNI, Springer, etc.\,pp.
    }
    \item{
        In Fußnoten schreiben Sie ausschließlich weiterführende Informationen.
        \begin{itemize}
            \item Bei URLs geben Sie immer das Datum des letzten erfolgreichen Zugriffs an\footnote{Der Grund dahinter ist, dass es sein kann, dass die von Ihnen verlinkte Webseite zwischenzeitlich gegenteilige Inhalte aufweisen oder komplett vom Netz genommen sein könnte. Durch die Datumsangabe ist es den Begutachtenden möglich, im Internet Archive oder in der Wayback Machine den von Ihnen verwendeten Stand zu reproduzieren.}.
            \item Bei Internet-\enquote{Quellen} verlinken Sie die Hauptseite einmal! Nicht für jede Unterseite eine neue Fußnote anlegen!
        \end{itemize}
    }
    \item{
        Fußnoten setzen Sie einmal. Sollten Sie inhaltlich das gleiche später noch einmal benötigen, so verwenden Sie die gleiche Fußnote wieder, selbst wenn sie mehrere Seiten zurückliegen sollte. Hier als Beispiel die Wiederverwendung der Fußnote mit dem Leitfaden\footnoteref{fn:Leitfaden}\textsuperscript{,}\footnote{Wiederverwendete Fußnoten sind im PDF anklickbar und führen Lesende direkt zur entsprechenden Stelle im Dokument. Sie erstellen sie, indem Sie Ihrer Fußnote ein \inlinecode{{\textbackslash}label} geben und dann später mittels \inlinecode{{\textbackslash}footnoteref} referenzieren.}.
    }
\end{itemize}

\comment{Die in den Wirtschafts- und Geisteswissenschaften weit verbreitete Praxis, Quellen und Referenzen in den Fußnoten aufzuführen hat auch seine guten Gründe und Vorteile. Wenn Sie einzelne Seiten aus einem Werk kopieren sind sofort Primärtext und Quellenangaben auf einem Blatt. Die Kopie ist sofort verwendbar. -- In den Ingenieruwissenschaften müssen Sie zusätzlich zum Primärtext auch das Literaturverzeichnis kopieren damit die Kopie vollständig ist.}

Beachten Sie, dass bei doppelseitigem Druck neue Kapitel immer auf einer rechten Seite beginnen, auch wenn das zu einer leeren linken Seite davor führt.

Beim doppelseitigen Druck ist der innere Rand der Seiten schmaler als der äußere, da nach der Bindung der gemeinsame innere\footnote{Abstand des rechten Randes des Textes der linken Seite zum linken Rand des Textes der rechten Seite} Rand optisch genau so groß sein soll, wie die äußeren Ränder. --\,Bei den hier gewählten Rändern haben Sie außen einen größeren Korrekturrand. Dieser ist notwendig, da der in dieser Vorlage voreingestellte Zeilenabstand den Begutachtenden keine Kommentierung im Fließtext erlaubt.

Wie Sie sicherlich bemerkt haben, ist diese Vorlage so konfiguriert, dass Absätze einen kleinen vertikalen Abstand zueinander haben. Die erste Zeile jedes Absatzes ist \emph{nicht} eingerückt. Dieses Verhalten können Sie in der Datei \texttt{config.tex} anpassen:
\begin{itemize}
    \item \inlinecode{\textbackslash setlength\{\textbackslash parindent\}\{0pt\}} --\,definiert den Zeilenvorschub, hier auf Null, und
    \item \inlinecode{\textbackslash setlength\{\textbackslash parskip\}\{4pt\}} --\,definiert die Absatzdistanz, hier auf 4 Punkte.
\end{itemize}

Die Anweisung \inlinecode{\textbackslash noindent} zu Beginn eines Absatzes unterdrückt den Zeilenvorschub. Das ist von Relevanz, falls Sie einen Wert ungleich Null für das \inlinecode{\textbackslash parindent} definiert haben. Es ist nämlich Standardeinstellung von \LaTeX{}, dass der erste Absatz nach der Überschrift des Kapitels bzw. der Sektion nicht eingerückt wird, alle anderen aber schon. Das ist unschön, wenn Sie ein Blockelement im Fließtext haben, wie zum Beispiel zu Beginn dieses Kapitels nach dem Kommentar. Wie Sie im Quelltext sehen können, ist an der entsprechenden Stelle ein \inlinecode{\textbackslash noindent} platziert. Beobachten Sie das Verhalten, wenn Sie in der \texttt{config.tex} den Wert für \inlinecode{\textbackslash parindent} verändern!

Apropos Absätze: {\color{red}Absatzumbrüche erzeugen Sie in \LaTeX\ immer mit einem doppelten Zeilenumbruch im Quelltext.} Achten Sie also drauf, wo Sie einzelne und doppelte Zeilenumbrüche im Quelltext verwenden. Erzeugen Sie keine Absatzumbrüche durch doppelte \LaTeX{}-Zeilenumbrüche (\inlinecode{\textbackslash\textbackslash\textbackslash\textbackslash}) oder entsprechende Ersatzkommandos wie \texttt{\textbackslash newline}. Überlegen Sie auch genau, wann Sie Seitenumbrüche (\inlinecode{\textbackslash clearpage} und \inlinecode{\textbackslash cleardoublepage}) verwenden.

Gerne verwenden Studierende für wichtige Begriffe einen \Gls{glossar}. Das ist recht einfach und wird in dieser Vorlage mit dem \enquote{\texttt{glossaries}}-Paket umgesetzt. Alle Glossareinträge befinden sich in der Datei \texttt{frontbackmatter\textbackslash Glossary.tex} und sollten dort von Ihnen verwaltet werden. Sie werden mit der Anweisung \inlinecode{\textbackslash Gls\{Eintrag\}} referenziert.

Über das \enquote{glossaries}-Paket können Sie auch \Glspl{abk} definieren. Auf die definierten \Glspl{abk} greifen Sie dann zu mit
\begin{itemize}
    \item \inlinecode{\textbackslash acrshort\{abk\}}: erzeugt nur die Abkürzung, hier also \enquote{\acrshort{abk}},
    \item \inlinecode{\textbackslash acrlong\{abk\}}: erzeugt das Ausgeschriebene, hier also \enquote{\acrlong{abk}}, sowie
    \item \inlinecode{\textbackslash acrfull\{abk\}}: erzeugt beides, hier also \enquote{\acrfull{abk}}.
\end{itemize}

Außerdem gibt es, wie Sie sicherlich gemerkt haben, auch ein paar Tricks für die korrekte Referenzierung des Plurals. Dazu verwenden Sie die Anweisung \inlinecode{\textbackslash Glspl\{abk\}}. Beachten Sie, dass beim erstmaligen Aufruf (siehe oben) das Verhalten von \texttt{acrfull} verwendet wird.

Näheres zum Umgang mit dem \enquote{glossaries}-Paket finden Sie in der Overleaf-Hilfe unter \url{https://www.overleaf.com/learn/latex/Glossaries}.

Ein letzter Stilhinweis: Vermeiden Sie nach Möglichkeit Silben-, Wort-, Satz- und Absatzwaisen. D.h., dass Sie vermeiden eine neue Zeile oder gar eine ansonsten leere Seite mit nur einer Silbe oder einem einzelnen Wort zu beginnen. Versuchen Sie, Seiten mit weniger als $5$ Zeilen zu vermeiden.

Apropos Silben: Bei langen Kompositen und an manchen Zeilenenden tut sich \LaTeX{} sehr schwer mit der Silbentrennung. Sie haben mehrere Optionen, wie Sie dann vorgehen können:
\begin{itemize}
    \item{Zwingen Sie das ganze Wort in die neue Zeile indem Sie eine \texttt{mbox} um das Wort legen. Das ist auch sinnvoll, wenn Sie Gesetze o.\,ä. referenzieren und sie zusammenbehalten wollen: \inlinecode{\textbackslash mbox\{§§ 126, 126a BGB\}}.}
    \item{Sie können \LaTeX{} explizit die Silbenstrennung vorgeben, indem Sie die Trennstellen markieren: \inlinecode{Do\textbackslash-nau\textbackslash-dampf\textbackslash-schiff\textbackslash-fahrt}. Bei häufig verwendeten Wörtern definieren Sie die Silbentrennung lieber am Ende des Vorspanns mit der \texttt{hyphenation}-Umgebung (in dieser Vorlage in der \texttt{thesis.tex} in Zeile $47$). Aber Vorsicht: \LaTeX{} wird die Wörter dann \emph{nur} an diesen Stellen trennen.}
    \item{Nicht zu trennende Wörter können Sie statt mit einer \texttt{mbox} auch mit einer Tilde (geschütztes Leerzeichen) binden: \inlinecode{§§\textasciitilde 126,\textasciitilde 126a\textasciitilde BGB}.}
    \item{Das geschützte halbe Leerzeichen verwenden Sie bei Abkürzungen wie \inlinecode{z.\textbackslash,B.}.}
\end{itemize}

Der Rest dieses Werkes ist lediglich ein Platzhalter für die grundständige Struktur einer ingenieurwissenschaftlichen Ausarbeitung und gibt dabei ein paar Einblicke in den Textsatz mit dieser \LaTeX{}-Vorlage\footnote{Eine Word-Vorlage wurde erst einmal nicht erstellt. Sollten Sie Word bevorzugen, folgen Sie sinngemäß dieser Vorlage!}\textsuperscript{,}\footnote{Im Allgemeinen finden Sie in den Kapiteln Hinweise auf die jeweils geforderten Inhalte.}. Befolgen Sie mit Ausnahme der hier folgenden Hinweise ansonsten den eingangs dieses Kapitels erwähnten  Leitfaden\footnoteref{fn:Leitfaden}.


\section{Motivation}\label{sec:Intro:Motivation}
Beschreiben Sie die Motivation hinter der Themenstellung. Weshalb ist es wichtig, dass Sie das Thema wissenschaftliche bearbeiten und quantitativ belastbare Ergebnisse produzieren?


\section{Ziel der Arbeit}\label{sec:Intro:Goal}
Beschreiben Sie kurz und prägnant das Ziel Ihrer Arbeit. Sollten Sie Thesen haben, wäre hier ein guter Ort, um sie erstmals aufzuführen.

\comment{Beachten Sie unbedingt die Hinweistexte in der \texttt{config.tex}, beispielsweise zum Auftragsblatt (siehe \hyperlink{taskpage}{Seite }\pageref{taskpage}) oder dem Einsatz von Bib\LaTeX\ anstelle von Bib\TeX.}


\section{Gliederung}\label{sec:Intro:Structure}
Beschreiben Sie die Gliederung Ihrer Abschlussarbeit, damit Lesende einen schnellen Überblick über den Inhalt bekommen. Beispiel:

\emph{Der Rest dieser Ausarbeitung ist entsprechend der Mehrheit ingenieurwissenschaftlicher Arbeiten gegliedert. Nach einer Übersicht der notwendigen Grundlagen und der wichtigsten verwandten Arbeiten im nächsten Kapitel folgt ab Seite~\pageref{chap:Concept} die eigene Konzeption eines Lösungsansatzes. Die Umsetzbarkeit dessen wird in \autoref{chap:ProofOfConcept} ab Seite~\pageref{chap:ProofOfConcept} dargestellt. Die auf Basis der prototypischen Umsetzung durchgeführte Evaluation schließt sich ab Seite~\pageref{chap:Evaluation} an. Die Auswertung der Evaluation erfolgt im \autoref{chap:Results} \enquote{\nameref{chap:Results}} bevor die zusammenfassende Diskussion die Ausarbeitung abschließt.}