\chapter{Appendix Test}\label{apdx:A}
\section{Tabellen}\label{sec:apdx:Tables}
\begin{table}[hbt]
	\begin{center}
		\begin{tabular}{l|cc|cc|cc|cc}
			&
			\multicolumn{2}{c|}{\small\textbf{2013/06}} &
			\multicolumn{2}{c|}{\small\textbf{2014/06}} &
			\multicolumn{2}{c|}{\small\textbf{2015/04}} &
			\multicolumn{2}{c}{\small\textbf{2015/06}} \\

			{\small\textbf{Parameter}} &
			{\small\textbf{$\overline{X}^{13|6}$}} &
			{\small\textbf{$\sigma_X^{13|6}$}} &
			{\small\textbf{$\overline{X}^{14|6}$}} &
			{\small\textbf{$\sigma_X^{14|6}$}} &
			{\small\textbf{$\overline{X}^{15|4}$}} &
			{\small\textbf{$\sigma_X^{15|4}$}} &
			{\small\textbf{$\overline{X}^{15|5}$}} &
			{\small\textbf{$\sigma_X^{15|6}$}} \\
            \midrule

			{Alter} &
			$20.6$ & $2.26$ &
			$20.5$ & $2.23$ &
			$20.6$ & $2.27$ &
			$20.4$ & $2.29$ \\

			{Geschlecht$^{\ast}$} &
			$.93$ & n/a &
			$.92$ & n/a &
			$.88$ & n/a &
			$.86$ & n/a \\

			{Zugehörigkeit$^{\oplus}$} &
			$.06$ & n/a &
			$.03$ & n/a &
			$.02$ & n/a &
			$.01$ & n/a \\

			{Durchgang$^{\mp}$} &
			$1.16$ & $.42$ &
			$1.14$ & $.37$ &
			$1.21$ & $.48$ &
			$1.19$ & $.44$ \\

			{Pflichtmodul$^{\bigtriangledown}$} &
			$.36$ & n/a &
			$.33$ & n/a &
			$.31$ & n/a &
			$.31$ & n/a \\

			{Prüfung$^{\bigtriangledown}$} &
			$.32$ & n/a &
			$.29$ & n/a &
			$.30$ & n/a &
			$.26$ & n/a \\

			{Interesse$^{\bigtriangledown}$} &
			$.75$ & n/a &
			$.78$ & n/a &
			$.73$ & n/a &
			$.72$ & n/a \\
			
			\midrule
			\multicolumn{4}{l}{$\ast$: $0$\ldots weiblich, $1$\ldots männlich} &
			\multicolumn{5}{r}{$\oplus$: $0$\ldots TU~Dresden, $1$\ldots abdere} \\
			\multicolumn{6}{l}{$\mp$: $1$\ldots Erster, $2$\ldots Zweiter, $3$\ldots Dritter oder mehr} &
			\multicolumn{3}{r}{$\bigtriangledown$: $0$\ldots ja, $1$\ldots nein}
		\end{tabular}
	\end{center}
	\vspace{-1em}
	\caption[Komposition der Studierendengruppe]{Komposition der Studierendengruppe im betrachteten Setup}\label{tab:Students}
\end{table}

\lipsum[10]

\begin{mytable}
	\begin{center}
		\begin{longtable}{c|p{1.7cm}|ccccc|cc|cc|c}
			\multicolumn{2}{c|}{} & \multicolumn{5}{c|}{\textbf{Diagramm}} & \multicolumn{2}{c|}{\textbf{Prüfung \&}} & \multicolumn{2}{c|}{} & \\
			\multicolumn{2}{c|}{} & \multicolumn{3}{c}{\textbf{UML}} & \multicolumn{2}{c|}{\textbf{andere}} & \multicolumn{2}{c|}{\textbf{Unterstützung}} & \multicolumn{2}{c|}{\textbf{Analyse}} & \\
			{\rotatebox{90}{\parbox{1.5cm}{\textbf{Übung}}}} & \textbf{Aufgaben\-thema} & {\rotatebox{90}{\parbox{1.5cm}{\raggedright\small\textbf{Topologie}}}} & {\rotatebox{90}{\parbox{1.5cm}{\raggedright\small\textbf{Sequenz}}}} & {\rotatebox{90}{\parbox{1.5cm}{\raggedright\small\textbf{Zustand}}}} & {\rotatebox{90}{\parbox{1.5cm}{\raggedright\small\textbf{Schema}}}} & {\rotatebox{90}{\parbox{1.5cm}{\raggedright\small\textbf{Graph}}}} & {\rotatebox{90}{\parbox{1.5cm}{\raggedright\small\textbf{Formel}}}} & {\rotatebox{90}{\parbox{1.5cm}{\raggedright\small\textbf{Tabelle}}}} & {\rotatebox{90}{\parbox{1.5cm}{\raggedright\small\textbf{DIY-\newline Beispiel}}}} & {\rotatebox{90}{\parbox{1.5cm}{\raggedright\small\textbf{Wireshark-Trace}}}} & {\rotatebox{90}{\parbox{1.5cm}{\raggedright\small\textbf{Prosa}}}} \\
			\toprule
			\endfirsthead

			\multicolumn{12}{l}{{$\leadsto$\,fortgesetzt von vorheriger Seite.}} \\
			\toprule
			\multicolumn{2}{c|}{} & \multicolumn{5}{c|}{\textbf{Diagramm}} & \multicolumn{2}{c|}{\textbf{Prüfung \&}} & \multicolumn{2}{c|}{} & \\
			\multicolumn{2}{c|}{} & \multicolumn{3}{c}{\textbf{UML}} & \multicolumn{2}{c|}{\textbf{andere}} & \multicolumn{2}{c|}{\textbf{Unterstützung}} & \multicolumn{2}{c|}{\textbf{Analyse}} & \\
			{\rotatebox{90}{\parbox{1.5cm}{\textbf{Übung}}}} & \textbf{Aufgaben\-thema} & {\rotatebox{90}{\parbox{1.5cm}{\raggedright\small\textbf{Topologie}}}} & {\rotatebox{90}{\parbox{1.5cm}{\raggedright\small\textbf{Sequenz}}}} & {\rotatebox{90}{\parbox{1.5cm}{\raggedright\small\textbf{Zustand}}}} & {\rotatebox{90}{\parbox{1.5cm}{\raggedright\small\textbf{Schema}}}} & {\rotatebox{90}{\parbox{1.5cm}{\raggedright\small\textbf{Graph}}}} & {\rotatebox{90}{\parbox{1.5cm}{\raggedright\small\textbf{Formel}}}} & {\rotatebox{90}{\parbox{1.5cm}{\raggedright\small\textbf{Tabelle}}}} & {\rotatebox{90}{\parbox{1.5cm}{\raggedright\small\textbf{DIY-\newline Beispiel}}}} & {\rotatebox{90}{\parbox{1.5cm}{\raggedright\small\textbf{Wireshark-Trace}}}} & {\rotatebox{90}{\parbox{1.5cm}{\raggedright\small\textbf{Prosa}}}} \\
			\toprule
			\endhead

			\bottomrule
			\multicolumn{12}{r}{Fortsetzung auf nächster Seite\,$\leadsto$} \\
			\endfoot

			\multicolumn{12}{c}{\tiny} \\
			\caption[Mehrseitige Tabelle]{Beispiel für eine mehrseitige Tabelle mit \enquote{longtable}-Umgebung.}\label{tab:TopicsRN}
			\endlastfoot

			\multirow{4}{*}{\rotatebox{90}{$\leftarrow$~U01}} & {\raggedright Computer\-netzwerk\-strukturen} & $\bullet$ & & & & & $\bullet$ & $\bullet$ & & & \\
			\cline{2-12}
			& {\raggedright Dienste und Protokolle} & & $\bullet$ & & $\bullet$ & & & $\bullet$ & & & $\bullet$ \\
			\cline{2-12}
			& {\raggedright Dienst\-elemente} & & $\bullet$ & $\bullet$ & & & & $\bullet$ & & & \\
			\cline{2-12}
			& {\raggedright OSI-Referenz\-modell} & & $\bullet$ & & $\bullet$ & & $\bullet$ & & & & \\
			\midrule

			\multirow{5}{*}{\rotatebox{90}{$\leftarrow$~U02}} & {\raggedright Nyquist-Theorem} & & & & & & $\bullet$ & & & & \\
			\cline{2-12}
			& {\raggedright PCM} & & & & $\bullet$ & & $\bullet$ & & & & \\
			\cline{2-12}
			& {\raggedright Modulation} & & & & $\bullet$ & & & $\bullet$ & & & \\
			\cline{2-12}
			& {\raggedright Line-Encoding} & & & & $\bullet$ & & $\bullet$ & & & & $\bullet$ \\
			\cline{2-12}
			& {\raggedright Multiplexing} & & & & & & $\bullet$ & $\bullet$ & & & $\bullet$ \\
			\midrule

			\multirow{4}{*}{\rotatebox{90}{$\leftarrow$~U03}} & {\raggedright Ethernet} & $\bullet$ & $\bullet$ & & $\bullet$ & & $\bullet$ & & & & \\
			\cline{2-12}
			& {\raggedright Switches} & $\bullet$ & & & & & $\bullet$ & & & & \\
			\cline{2-12}
			& {\raggedright transparente Bridges} & $\bullet$ & & & $\bullet$ & & & & & & $\bullet$ \\
			\cline{2-12}
			& {\raggedright Wireless-LAN} & & & & & & $\bullet$ & & & & \\
			\midrule

			\multirow{6}{*}{\rotatebox{90}{$\leftarrow$~U04}} & {\raggedright WiMAX} & & & & & & & $\bullet$ & & & $\bullet$ \\
			\cline{2-12}
			& {\raggedright RPR} & $\bullet$ & & & & & & & & & \\
			\cline{2-12}
			& {\raggedright Carrier-Ethernet} & $\bullet$ & & & $\bullet$ & & & & & & $\bullet$ \\
			\cline{2-12}
			& {\raggedright MPLS} & $\bullet$ & & & $\bullet$ & & & $\bullet$ & & & \\
			\cline{2-12}
			& {\raggedright SONET, SDH, OTN} & & & & $\bullet$ & & $\bullet$ & & & & \\
			\cline{2-12}
			& {\raggedright VPN} & $\bullet$ & & $\bullet$ & & & & & $\bullet$ & & $\bullet$ \\
			\midrule

			\multirow{5}{*}{\rotatebox{90}{$\leftarrow$~U05}} & {\raggedright Rahmen\-bildung} & & & & $\bullet$ & & & $\bullet$ & & & $\bullet$ \\
			\cline{2-12}
			& {\raggedright ECC} & & & & $\bullet$ & & $\bullet$ & $\bullet$ & & & \\
			\cline{2-12}
			& {\raggedright Paritäts\-bit} & & & & $\bullet$ & & & $\bullet$ & & & \\
			\cline{2-12}
			& {\raggedright CRC} & & $\bullet$ & & & & $\bullet$ & & & & \\
			\cline{2-12}
			& {\raggedright Stop-and-Wait-Protokoll} & & $\bullet$ & & $\bullet$ & & & & & & $\bullet$ \\
			\midrule

			\multirow{5}{*}{\rotatebox{90}{$\leftarrow$~U06}} & {\raggedright OSPF} & $\bullet$ & & & & $\bullet$ & $\bullet$ & $\bullet$ & & & \\
			\cline{2-12}
			& {\raggedright IP-Pakete} & & & & $\bullet$ & & & & & $\bullet$ & $\bullet$ \\
			\cline{2-12}
			& {\raggedright IP:\newline Adressen \& Subnetze} & $\bullet$ & & & $\bullet$ & & $\bullet$ & & & & \\
			\cline{2-12}
			& {\raggedright Routing} & $\bullet$ & $\bullet$ & & $\bullet$ & & & $\bullet$ & & & \\
			\cline{2-12}
			& {\raggedright Überlast\-kontrolle} & & $\bullet$ & & $\bullet$ & & & & & & $\bullet$ \\
			\midrule

			\multirow{4}{*}{\rotatebox{90}{$\leftarrow$~U07}} & {\raggedright Dienste: QoS} & & $\bullet$ & & & & & & & & \\
			\cline{2-12}
			& {\raggedright Schiebe\-fenster\-protokoll} & & $\bullet$ & & $\bullet$ & & & $\bullet$ & & & \\
			\cline{2-12}
			& {\raggedright TCP} & & $\bullet$ & $\bullet$ & & & & $\bullet$ & & $\bullet$ & \\
			\cline{2-12}
			& {\raggedright Protokoll\-vergleich} & & & & & & & $\bullet$ & & & \\
			\midrule

			\multirow{5}{*}{\rotatebox{90}{$\leftarrow$~U08}} & {\raggedright Ethernet-Performance} & & & & $\bullet$ & & $\bullet$ & $\bullet$ & & & \\
			\cline{2-12}
			& {\raggedright Fairness} & & $\bullet$ & & & & & & & & $\bullet$ \\
			\cline{2-12}
			& {\raggedright Überlast\-kontrolle} & & & & $\bullet$ & & $\bullet$ & $\bullet$ & & & \\
			\cline{2-12}
			& {\raggedright Performance-Steigerung} & & & & $\bullet$ & & $\bullet$ & & & & \\
			\cline{2-12}
			& {\raggedright DTN} & $\bullet$ & & & $\bullet$ & & & & & & \\
			\midrule

			\multirow{5}{*}{\rotatebox{90}{$\leftarrow$~U09}} & {\raggedright DNS} & $\bullet$ & & & $\bullet$ & & & $\bullet$ & & & $\bullet$ \\
			\cline{2-12}
			& {\raggedright E-Mail} & & & & & & & & & & $\bullet$ \\
			\cline{2-12}
			& {\raggedright WWW} & & & & $\bullet$ & & & & & $\bullet$ & $\bullet$ \\
			\cline{2-12}
			& {\raggedright SNMP} & & $\bullet$ & & & & & $\bullet$ & & & \\
			\cline{2-12}
			& {\raggedright Zusammen\-spiel von UDP, TCP, HTTP und DNS} & & $\bullet$ & & $\bullet$ & & & $\bullet$ & & & \\
			\midrule

			\multirow{5}{*}{\rotatebox{90}{$\leftarrow$~U10}} & {\raggedright Kodierung} & & & & $\bullet$ & & $\bullet$ & $\bullet$ & & & \\
			\cline{2-12}
			& {\raggedright VoIP} & $\bullet$ & & & $\bullet$ & & & $\bullet$ & & & $\bullet$ \\
			\cline{2-12}
			& {\raggedright RTP} & & & & $\bullet$ & & & & & $\bullet$ & \\
			\cline{2-12}
			& {\raggedright VoIP-Codecs} & & $\bullet$ & & $\bullet$ & & $\bullet$ & & & & \\
			\cline{2-12}
			& {\raggedright Video\-konferenz\-systeme} & $\bullet$ & & & & & $\bullet$ & & & & \\
			\midrule

			\multirow{4}{*}{\rotatebox{90}{$\leftarrow$~U11}} & {\raggedright RPC} & $\bullet$ & $\bullet$ & & & & & & & & $\bullet$ \\
			\cline{2-12}
			& {\raggedright verteilte Transaktionen} & $\bullet$ & $\bullet$ & $\bullet$ & & & & & & & $\bullet$ \\
			\cline{2-12}
			& {\raggedright Web-Services} & & & & & & & & & $\bullet$ & $\bullet$ \\
			\cline{2-12}
			& {\raggedright WebDAV} & & $\bullet$ & & & & & & & $\bullet$ & \\
			\midrule

			\multirow{3}{*}{\rotatebox{90}{$\leftarrow$~U12}} & {\raggedright mobile Kommuni\-kations\-netzwerke} & & & & $\bullet$ & & & & & & $\bullet$ \\
			\cline{2-12}
			& {\raggedright DHCP} & & $\bullet$ & $\bullet$ & & & & $\bullet$ & & & $\bullet$ \\
			\cline{2-12}
			& {\raggedright Mobile-IP} & $\bullet$ & $\bullet$ & & $\bullet$ & & & $\bullet$ & & & \\
			\midrule
			\multicolumn{2}{r|}{Summe} & $18$ & $20$ & $5$ & $32$ & $1$ & $20$ & $24$ & $1$ & $6$ & $21$ \\
			\multicolumn{2}{r|}{Anteil} & $.327$ & $.364$ & $.091$ & $.582$ & $.018$ & $.364$ & $.436$ & $.018$ & $.109$ & $.382$ \\
			\cline{3-7}
			\multicolumn{2}{c}{} & \multicolumn{5}{c}{$76$ Diagramme,} &
 \multicolumn{5}{c}{} \\
			\multicolumn{2}{c}{} & \multicolumn{5}{c}{davon $43$ UML ($56.6\%$)} &
 \multicolumn{5}{c}{} \\
		\end{longtable}
	\end{center}
	\vspace{-1.5em}
\end{mytable}


\begin{mytable}\begin{sidewaystable}
	\begin{center}
		\begin{tabular}{cc|cc|cc|cc|cc|cc|cc|cc|cc}
			\multicolumn{2}{c|}{} &
			\multicolumn{2}{c|}{\rotatebox[origin=c]{270}{\parbox{1.3cm}{\raggedright\small\textbf{before first reading}}}} &
			\multicolumn{2}{c|}{\rotatebox[origin=c]{270}{\parbox{1.3cm}{\raggedright\small\textbf{after first reading}}}} &
			\multicolumn{2}{c|}{\rotatebox[origin=c]{270}{\parbox{1.3cm}{\raggedright\small\textbf{current reading}}}} &
			\multicolumn{2}{c|}{\rotatebox[origin=c]{270}{\parbox{1.3cm}{\raggedright\small\textbf{before first tutorial}}}} &
			\multicolumn{2}{c|}{\rotatebox[origin=c]{270}{\parbox{1.3cm}{\raggedright\small\textbf{after first tutorial}}}} &
			\multicolumn{2}{c|}{\rotatebox[origin=c]{270}{\parbox{1.3cm}{\raggedright\small\textbf{current tutorial}}}} &
			\multicolumn{2}{c|}{\rotatebox[origin=c]{270}{\parbox{1.3cm}{\raggedright\small\textbf{reading \mbox{$\downarrow\bigtriangledown$\hspace*{.35cm}} tutorial}}}} &
			\multicolumn{2}{c}{\rotatebox[origin=c]{270}{\parbox{1.3cm}{\raggedright\small\textbf{tutorial \mbox{$\downarrow\bigtriangledown$\hspace*{.35cm}} reading}}}} \\

			\multicolumn{2}{c|}{data source} &
			{\small\textbf{$\overline{X}$}} &
			{\small\textbf{$\sigma_X$}} &
			{\small\textbf{$\overline{X}$}} &
			{\small\textbf{$\sigma_X$}} &
			{\small\textbf{$\overline{X}$}} &
			{\small\textbf{$\sigma_X$}} &
			{\small\textbf{$\overline{X}$}} &
			{\small\textbf{$\sigma_X$}} &
			{\small\textbf{$\overline{X}$}} &
			{\small\textbf{$\sigma_X$}} &
			{\small\textbf{$\overline{X}$}} &
			{\small\textbf{$\sigma_X$}} &
			{\small\textbf{$\overline{X}$}} &
			{\small\textbf{$\sigma_X$}} &
			{\small\textbf{$\overline{X}$}} &
			{\small\textbf{$\sigma_X$}} \\
			\midrule

			\multirow{3}{*}{15-G1}
			& 15-A1 & $6.38$ & $2.09$ & $5.55$ & $2.13$ & $5.99$ & $2.09$ & $6.39$ & $1.99$ & $6.02$ & $1.98$ & $6.05$ & $2.65$ & $1.89$ & $.62$ & $2.33$ & $.67$ \\
			& 15-A2 & \multicolumn{2}{c|}{not asked} & \multicolumn{2}{c|}{not asked} & 5.97 & 2.21 & \multicolumn{2}{c|}{not asked} & \multicolumn{2}{c|}{not asked} & $6.22$ & $2.25$ & $1.88$ & $.64$ & $2.36$ & $.71$ \\
			& 15-A3 & \multicolumn{2}{c|}{not asked} & \multicolumn{2}{c|}{not asked} & 5.91 & 2.08 & \multicolumn{2}{c|}{not asked} & \multicolumn{2}{c|}{not asked}& $6.27$ & $2.34$ & $1.86$ & $.69$ & $2.44$ & $.76$ \\

			\midrule
			\multirow{3}{*}{15-G2}
			& 15-A1 & $6.24$ & $2.37$ & $5.44$ & $2.20$ & $6.02$ & $2.31$ & $6.12$ & $2.04$ & $6.14$ & $2.14$ & $6.13$ & $1.98$ & $1.88$ & $.59$ & $2.31$ & $.70$ \\
			& 15-A2 & \multicolumn{2}{c|}{not asked} & \multicolumn{2}{c|}{not asked} & 5.98 & 2.16 & \multicolumn{2}{c|}{not asked} & \multicolumn{2}{c|}{not asked} & $6.20$ & $1.97$ & $1.86$ & $.61$ & $2.33$ & $.71$ \\
			& 15-A3 & \multicolumn{2}{c|}{not asked} & \multicolumn{2}{c|}{not asked} & 5.93 & 1.98 & \multicolumn{2}{c|}{not asked} & \multicolumn{2}{c|}{not asked}& $6.30$ & $2.00$ & $1.84$ & $.60$ & $2.37$ & $.77$ \\

			\midrule
			\multirow{3}{*}{15-G3}
			& 15-A1 & $6.29$ & $2.02$ & $5.22$ & $2.10$ & $5.91$ & $2.06$ & $6.42$ & $2.08$ & $5.89$ & $2.51$ & $6.01$ & $2.00$ & $1.88$ & $.59$ & $2.42$ & $.79$ \\
			& 15-A2 & \multicolumn{2}{c|}{not asked} & \multicolumn{2}{c|}{not asked} & 5.93 & 2.27 & \multicolumn{2}{c|}{not asked} & \multicolumn{2}{c|}{not asked} & $6.01$ & $1.99$ & $1.87$ & $.66$ & $2.34$ & $.74$ \\
			& 15-A3 & \multicolumn{2}{c|}{not asked} & \multicolumn{2}{c|}{not asked} & 5.90 & 2.15 & \multicolumn{2}{c|}{not asked} & \multicolumn{2}{c|}{not asked}& $6.00$ & $2.11$ & $1.84$ & $.72$ & $2.37$ & $.76$ \\

			\midrule
			15-CG & $\ast$ & \multicolumn{16}{c}{insufficient data (pool size $9$)} \\

			\midrule
			\multicolumn{18}{l}{$\bigtriangledown$: $1$\ldots yes, $2$\ldots neither, $3$\ldots no} \\

		\end{tabular}
		\caption[Seitwärtstabelle]{mit der \enquote{sidewaystable}-Umgebung drehen Sie Tabellen so, dass sie \enquote{nach außen} zeigen. --\,Die Zellen in der Titelzeile müssen Sie ggf. für linke Seiten ($90^{\circ}$ und \textbackslash{}raggedleft) und --\,wie hier\,-- rechte Seiten ($270^{\circ}$ und \textbackslash{}raggedright) anpassen.}\label{tab:Motivation2015}
	\end{center}
\end{sidewaystable}\end{mytable}



\section{Listen}\label{sec:apdx:Lists}
\noindent Mehrere Listentypen werden von dieser Vorlage unterstützt...

Punkt-Listen sind unsortierte Listen für Auflistungen mit
\begin{itemize}
    \item Punkten,    
    \item[--] kurzen Strichen,
    \item[$-$] langen Strichen,
    \item[$\ast$] Sternchen oder
    \item[$\blacksquare$] sonstigen Markern;
    \item{
        das geht auch in mehrere Ebenen 
        \begin{itemize}
            \item mit automatisch gesetzten Markern oder
            \item[$\alpha$] auch händisch definierten.
        \end{itemize}
    }
\end{itemize}

Sortierte Listen sind in der Regel Aufzählungen sortiert anhand
\begin{enumerate}
    \item arabischer Zahlen,
    \item[II.] römischer Zahlen oder
    % hier von Hand gesetzt, damit die unterschiedlichen Stile
    % im Dokument ohne Hacks direkt beieinander stehen können...
    \item[C.] Buchstaben;
    \setcounter{enumi}{3}
    \item{
        auch sie sind geschachtelt möglich, bspw.
        \begin{enumerate}
            \item Hund und
            \item Katze.
        \end{enumerate}
    }
\end{enumerate}

Beachten Sie, dass auch Listen immer vollständige Sätze mit Satzzeichen sein sollten. Das heißt, es gehört ein Komma hinter jedes Element bis auf die letzten beiden. Im vorletzten Element verwenden Sie eine Konjuktion\footnote{\enquote{und}, \enquote{sowie}, etc.pp.} oder eine Disjunktion\footnote{\enquote{oder}, \enquote{aber nicht}, etc.pp.}, sowie im letzten Element einen Punkt.

Mehr zu Listen und ihren Möglichkeiten finden Sie in der Overleaf-Hilfe:
\begin{center}
    \url{https://www.overleaf.com/learn/latex/Lists}
\end{center}

Manchmal wird es sehr hässlich, wenn Sie viele kurze Listenelemente haben. Das führt dazu, dass Sie einen \enquote{langen Schlauch} mit viel Whitespace rechts davon erhalten. Dann bietet es sich an, die Listen mehrspaltig setzen zu lassen:
\begin{multicols}{3}
\begin{itemize}
    \item erstes Element,
    \item zweites Element,
    \item drittes Element,
    \item viertes Element,
    \item fünftes Element,
    \item sechstes Element,
    \item siebtes Element,
    \item achtes Element sowie
    \item neuntes Element,
    \item zehntes Element,
    \item elftes Element,
    \item zwölftes Element,
    \item dreizehntes Element,
    \item vierzehntes Element,
    \item fünfzehntes Element,
    \item sechzehntes Element,
    \item siebzehntes Element und
    \item achtzehntes Element.
\end{itemize}
\end{multicols}

Wenn die Elemente länger werden, nutzen Sie weniger Spalten:
\begin{multicols}{2}
\begin{itemize}
    \item neunzehntes Element,
    \item zwanzigstes Element,
    \item einundzwanzigstes Element,
    \item zweiundzwanzigstes Element,
    \item dreiundzwanzigstes Element,
    \item vierundzwanzigstes Element,
    \item fünfundzwanzigstes Element,
    \item sechsundzwanzigstes Element,
    \item siebenundzwanzigstes Element,
    \item achtundzwanzigstes Element,
    \item neunundzwanzigstes Element und
    \item dreißigstes Element.
\end{itemize}
\end{multicols}
\noindent Es gibt in dieser Vorlage auch eine Beschreibungsumgebung:
\begin{description}
  \item[Zweiter Lipsum-Abschnitt:] \lipsum[11].
  \item[Erster Lipsum-Abschnitt:] \lipsum[12].
\end{description}


\section{Grafiken}\label{sec:apdx:Figures}
\subsection{Einfache Grafiken}\label{subsec:apdx:Figures:Simple}
Es gibt ein paar vorbereitete Abbildungsumgebungen in dieser Vorlage. Hier sind ein paar Beispiele; näheres entnehmen Sie der Datei \enquote{config.tex}.

% Parameterliste:
%   1. Breite der Abbildung
%   2. Dateiname ohne „figures“-Ordner und ohne Dateiendung
%       (mögliche Typen: PDF oder PNG)
%   3. Label zum Referenzieren der Abbildung;
%       „fig:“ wird automatisch ergänzt
%   4. Beschriftung der Abbildung im Inhaltsverzeichnis
%   5. Beschriftung der Abbildung im Dokument
\img{.5\linewidth}{setup}{AbbLabel1}{Beschriftung TOC}{Beschriftung Dokument}

% Parameterliste
%   1. Label des Containers
%   2. Beschriftung des Containers im Inhaltsverzeichnis
%   3. Beschriftung des Containers im Dokument
%   4. Breite der oberen Abbildung
%   5. obere Abbildung
%   6. Beschriftung der oberen Abbildung
%   7. Breite der unteren Abbildung
%   8. untere Abbildung
%   9. Beschriftung der unteren Abbildung
\imgABstack{AbbLabel2}{Gestapelte Abbildungen in einem Container}{Innerhalb der Beschriftungen können Sie referenzieren indem Sie \enquote{-A} und \enquote{-B} an das Label anhängen (\ref{fig:AbbLabel2-A}; \ref{fig:AbbLabel2-B}).}%
{.8\linewidth}{pdf_gaus_vs_uni_vs_10_40_160}{obere Abbildung}%
{.7\linewidth}{qq-plot_gaus_vs_160}{untere Abbildung}


\section{Theoreme}\label{sec:apdx:Theorem}
Diese Vorlage hat eine ganze Reihe Theorem-Umgebungen definiert. Hier ein paar Beispiele:
\begin{workingthesis}[Transitivität]\label{workingthesis:Transitivity}
	Alle technischen Eigenschaften von VMs lassen sich auf Cloud Container übertragen.
\end{workingthesis}

\begin{definition}[Werkzeug]\label{Def:GeneralTool}
	Jedes physische oder intellektuelle Objekt, mit dem eine Aufgabe erledigt werden kann, ist ein \textbf{Werkzeug}.
\end{definition}

\begin{corollary}[Software-Werkzeug]\label{Cor:WoftwareTool}
	Angenommen, die intellektuelle Eigenschaft bezieht sich auf einen in einer Programmiersprache implementierten Algorithmus, dann ist das kompilierte Programm ein \textbf{Software-Werkzeug}.
\end{corollary}

\begin{assertion}[\LaTeX{}-Kurs]
	Zum besten Verständnis dieser Vorlage nehmen wir an, dass Sie am \LaTeX{}-Kurs teilgenommen und dessen Inhalte verstanden haben.
\end{assertion}

Bisher haben Sie Beispiele gesehen für die Umgebungen \texttt{comment}, \texttt{definition}, \texttt{corollary}, \texttt{assertion} und \texttt{workingthesis}. Experimentieren Sie mit den anderen Umgebungen: \texttt{theorem}, \texttt{lemma}, \texttt{proof}, und \texttt{example}.

\clearpage
\section{Quellkode}\label{sec:apdx:Listings}
Mit dem \texttt{float}-Argument können Sie Ihren Quellkode an die Stelle des Auftretens (\texttt{float=h}), oben an den Seitenanfang (\texttt{float=t}) oder unten ans Ende der Seite (\texttt{float=b}) setzen.
\begin{lstlisting}[float=h,language=Pascal,frame=tb,caption={Beispielkode (Pascal)},label=lst:uselessPascal]
for i:=maxint downto 0 do
begin
{ do nothing }
end;
\end{lstlisting}

\begin{lstlisting}[float=h,language=Java,frame=tb,caption={Beispielkode (Java Swing)},label=lst:uselessJava]
package start;

import javax.swing.*;        

public class HelloWorldSwing {
    /**
     * GUI erzeugen und anzeigen.
     * Wegen der Thread-Sicherheit sollte diese Methode
     * vom Event-Dispatcher aufgerufen werden.
     */
    private static void createAndShowGUI() {
        JFrame frame = new JFrame("HelloWorldSwing");
        frame.setDefaultCloseOperation(JFrame.EXIT_ON_CLOSE);

        JLabel label = new JLabel("Hello World");
        frame.getContentPane().add(label);

        frame.pack();
        frame.setVisible(true);
    }

    /**
     * Hier wird ein Job fuer den Event-Dispatcher geplant.
     */
    public static void main(String[] args) {
        javax.swing.SwingUtilities.invokeLater(new Runnable() {
            public void run() {
                createAndShowGUI();
            }
        });
    }
}
\end{lstlisting}