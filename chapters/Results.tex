\chapter{Ergebnisse}\label{chap:Results}
Hier sollten Sie einerseits explizit Ihre Forschungsfragen beantworten, andererseits näher die Ergebnisse beleuchten. Falls eine Frage mit \enquote{Ja} beantwortet wurde, dann sollten Sie direkt die Frage \enquote{Wie gut?} beantworten. Falls eine Frage mit \enquote{Nein} beantwortet wurde, dann sollten Sie direkt die Frage \enquote{Weshalb nicht?} beantworten.

Sollten Sie anstatt mit Forschungsfragen mit Hypothesen in die Arbeit gegangen sein, sollten Sie spätestens hier die Bestätigung oder die Widerlegung führen. Im Falle einer Bestätigung ist wieder die Frage \enquote{Wie gut?} zu beantworten. Im Falle einer Widerlegung ist eine Antithese aufzustellen, welche mit etwas zusätzlicher Untersuchung (zweite Evaluation) zu einer Synthese führen sollte. Ansonsten im letzten Abschnitt des letzten Kapitels (Zusammenfassung $rarr$ Ausblick) als zukündtige Fortsetzungsmöglichkeit aufführen.