\chapter{ChatGPT und ähnliches}\label{apdx:ChatGPT}
\section{Allgemeines}
Als Studierender kann die Erstellung einer Abschlussarbeit eine Herausforderung sein, die viel Zeit und Anstrengung erfordert. ChatGPT\footnote{\url{https://chat.openai.com/chat} --\,zuletzt am 15. Februar 2023 erfolgreich abgerufen}, ein künstlicher Intelligenz-Chatbot, kann Ihnen bei dieser Aufgabe helfen.

Ein Vorteil ist, dass es eine Fülle von Informationen zu nahezu jedem Thema enthält, die für die Recherche Ihrer Arbeit nützlich sein können. Sie können ChatGPT Fragen stellen, um sich über das Thema Ihrer Arbeit zu informieren und wertvolle Einsichten und Informationen zu erhalten.

Ein weiterer Vorteil von ChatGPT ist, dass es Ihnen helfen kann, Ihre Arbeit zu strukturieren und zu organisieren. Sie können um Hilfe bitten, wenn Sie Schwierigkeiten haben, einen Überblick über Ihre Arbeit zu erstellen oder wenn Sie nicht sicher sind, wie Sie Ihre Argumentation am besten strukturieren können.

Allerdings gibt es auch einige Gefahren bei der Verwendung von ChatGPT für die Erstellung von Abschlussarbeiten. Es besteht die Möglichkeit, dass Sie zu sehr von der künstlichen Intelligenz abhängig werden und dadurch Ihre eigene kritische Denkfähigkeit vernachlässigen. Sie sollten ChatGPT daher als Werkzeug und nicht als Ersatz für Ihre eigene Denkarbeit verwenden.

Ein weiterer wichtiger Aspekt bei der Nutzung von ChatGPT für die Erstellung von Abschlussarbeiten ist das Urheberrecht. Sie sollten sicherstellen, dass Sie alle Quellen, die Sie aus ChatGPT zitieren, ordnungsgemäß zitieren\footnote{Anmerkung: ChatGPT \enquote{erfindet} Quellen. Sie müssen die tatsächlichen Quellen für Informationen in von ChatGPT generierten Texten recherchieren und dann anschließend ordentlich referenzieren.} und dass Sie keine urheberrechtlich geschützten Informationen aus ChatGPT verwenden, ohne die Genehmigung des Urhebers einzuholen.

Insgesamt kann ChatGPT ein sehr nützliches Werkzeug für die Erstellung von Abschlussarbeiten sein, solange Sie es verantwortungsvoll und als Ergänzung zu Ihrer eigenen Arbeit verwenden. Wenn Sie ChatGPT klug nutzen, kann es Ihnen dabei helfen, Ihre Arbeit schneller und effektiver zu erledigen und gleichzeitig ein besseres Verständnis für das Thema zu entwickeln.

\comment{Der obige Text wurde von ChatGPT auf als Antwort auf die Aufgabe \enquote{\emph{Erstelle einen Text, in dem Studierenden ChatGPT als Werkzeug für die Erstellung von Abschlussarbeiten erklärt wird. Gehe dabei auf Vorteile und Gefahren sowie das Urheberrecht ein.}} erzeugt.}

\section{Konsequenzen der Nutzung}
Wenn Sie ChatGPT und/oder ähnliche Werkzeuge verwenden, vergessen Sie nicht, diese in der Liste der Hilfsmittel (siehe Seite~\pageref{Tools}) anzugeben. Ohne diese Angabe handelt es sich bei Ihrer Abschlussarbeit um ein Plagiat beziehungsweise einen Täuschungsversuch.

\comment{Der folgende Text stammt von ChatGPT als Antwort auf die Aufgabe \enquote{emph{Beschreibe die juristischen Konsequenzen eines Täuschungsversuchs bei einer Abschlussarbeit.}}.}

Ein Täuschungsversuch bei einer Abschlussarbeit kann schwerwiegende juristische Konsequenzen haben. In den meisten akademischen Institutionen gilt das Täuschen bei der Erstellung von wissenschaftlichen Arbeiten als Verstoß gegen die akademischen Regeln und wird als akademisches Fehlverhalten betrachtet. Wenn ein Täuschungsversuch aufgedeckt wird, kann dies zur Aberkennung der Arbeit und zu weiteren Sanktionen führen, wie zum Beispiel der Verweigerung des Studienabschlusses, der Exmatrikulation oder sogar strafrechtlichen Konsequenzen.

In einigen Ländern kann das Täuschen bei einer Abschlussarbeit auch als Straftat gelten, die mit Geldstrafen oder sogar Gefängnisstrafen geahndet wird. Es ist daher von großer Bedeutung, dass Studierende bei der Erstellung von Abschlussarbeiten auf Integrität und Ehrlichkeit achten. Es ist wichtig, dass die Arbeit der eigenen geistigen Leistung entspricht und dass alle Quellen, die verwendet wurden, ordnungsgemäß zitiert und referenziert werden.

In vielen Fällen können Studierende von den Lehrenden und Bibliotheken der Institutionen Unterstützung bei der Vermeidung von Täuschungsversuchen und bei der Einhaltung der akademischen Regeln und Best Practices erhalten. Es ist ratsam, sich frühzeitig über die Anforderungen und Erwartungen im Zusammenhang mit der Erstellung einer Abschlussarbeit zu informieren und bei Bedarf Unterstützung und Beratung einzuholen.